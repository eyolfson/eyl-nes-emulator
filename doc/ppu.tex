% Copyright 2016 Jonathan Eyolfson
%
% This work is licensed under a Creative Commons Attribution-ShareAlike 4.0
% International License. You should have received a copy of the license along
% with this work. If not, see <http://creativecommons.org/licenses/by-sa/4.0/>.

\chapter{PPU}

Screen resolution is $256 \times 240$ pixels.
Tiles are $8 \times 8$, $8 \times 16$, or $16 \times 8$ pixels.
There are $64$ possible colours.

However, some of these colours are duplicates (black).

The PPU runs at $3 \times$ the frequency of the CPU.

Each PPU cycle (during the drawing phase) the PPU outputs one pixel to the
screen.

It has per pixel scrolling.

It can support up to $64$ sprites.

A scanline is one row of the screen, it renders line by line ($y$ is fixed, $x$
ranges over $0-256$).

On a single scanline, it only supports $8$ sprites.

If there are more than $8$ sprites, it will set a sprite overflow bit.

It can detect hits to sprite 0.

It is controlled by memory mapped registers to the CPU.

It has \texttt{2 KiB} of RAM.

It has \texttt{256 B} of RAM for sprites.

It has \texttt{64 B} of RAM for sprites on the current scanline.

Each sprite takes \texttt{4 B}.

\section{Memory}

The memory map is:

\hexadecimal{0000}: Tile Data 0

\hexadecimal{1000}: Tile Data 1

\hexadecimal{2000}: Nametable 0

\hexadecimal{2400}: Nametable 1

\hexadecimal{2800}: Nametable 2

\hexadecimal{2C00}: Nametable 3

\hexadecimal{3F00}: Palette RAM

\section{Object Attribute Memory (OAM)}

\section{Registers}

There are 8 registers accessible by the CPU.

\section{Drawing a Pixel}

Background is drawn from the current nametable.

\subsection{Tile}

A single $8 \times 8$ tile occupies $16$ bytes.
Consider the following tile data:

\binary{00000000}
\binary{00000000}
\binary{01000000}
\binary{01100000}

\binary{01110000}
\binary{01111000}
\binary{01111100}
\binary{01111110}

\binary{00011111}
\binary{00000001}
\binary{01000001}
\binary{01100001}

\binary{01010001}
\binary{01001000}
\binary{01000100}
\binary{01111110}

The first 8 bytes encode the low bit of the palette select.
The last 8 bytes encode the high bit of the palette select.
The pixels themselves are in row-major order.

\begin{tikzpicture}
  \matrix [matrix of nodes, nodes in empty cells,
           row sep=1mm, column sep=1mm,
           nodes=draw,
           minimum width=8mm, minimum height=8mm] {
      0 & 0 & 0 & 2 & 2 & 2 & 2 & 2 \\
      0 & 0 & 0 & 0 & 0 & 0 & 0 & 2 \\
      0 & 3 & 0 & 0 & 0 & 0 & 0 & 2 \\
      0 & 3 & 3 & 0 & 0 & 0 & 0 & 2 \\
      0 & 3 & 1 & 3 & 0 & 0 & 0 & 2 \\
      0 & 3 & 1 & 1 & 3 & 0 & 0 & 0 \\
      0 & 3 & 1 & 1 & 1 & 3 & 0 & 0 \\
      0 & 3 & 3 & 3 & 3 & 3 & 3 & 0 \\
  };
\end{tikzpicture}

\subsection{Palettes}

Held in RAM, maps the palette select to a colour.

\subsection{Nametable}

A nametable has $960$ bytes storing tile indices in row-major order.
The $960$ tiles make up the background ($32 \times 30$ tiles).
The remaining $64$ bytes stores attribute data.
The attribute data stores which palette to use.

\subsection{Background}

\textit{Relevant Internal Registers: \texttt{v}, \texttt{x}}

The current scanline is the $y$ coordinate.

A grid of the pixels is shown below:

\begin{tikzpicture}
  \draw (0, 0) -- (80mm, 0);
  \node [above] at (40mm, 0) {$x$};
  \node [above] at (0, 0) {\texttt{0}};
  \node [above] at (80mm, 0) {\texttt{255}};

  \draw (0, 0) -- (0, -70mm);
  \node [left] at (0, -35mm) {$y$};
  \node [left] at (0, 0) {\texttt{0}};
  \node [left] at (0, -70mm) {\texttt{239}};
\end{tikzpicture}

v: \hexadecimal{2000} = \binary{0010 NNYY YYYX XXXX}

v: \hexadecimal{2000} = \binary{0010 NN11 11YY YXXX}


Example, assume y scroll is  $64$ = \hexadecimal{40} = \binary{0100 0000}.

The tiles are...

\hexadecimal{2100} ... \hexadecimal{211F}

\hexadecimal{23A0} ... \hexadecimal{23BF}

...

\hexadecimal{2800} ... \hexadecimal{281F}

\hexadecimal{2820} ... \hexadecimal{283F}

\hexadecimal{2840} ... \hexadecimal{285F}

\hexadecimal{2860} ... \hexadecimal{287F}

\hexadecimal{2880} ... \hexadecimal{289F}

\hexadecimal{28A0} ... \hexadecimal{28BF}

\hexadecimal{28C0} ... \hexadecimal{28DF}

\hexadecimal{28E0} ... \hexadecimal{28FF}

\subsection{Sprite}

The OAM stores sprite data.
There is a maximum of 8 sprites per scanline.

\section{Vertical Blanking Interval}

Also known as: \texttt{VBLANK}

\section{Scrolling}

Internal registers: \texttt{t}, \texttt{v}, \texttt{x}.
